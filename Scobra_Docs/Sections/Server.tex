\chapter{The Server: Basics}

This section will guide users on how to setup an account on the Yale-NUS computer cluster and how to navigate in the server using the command line and using Jupyter notebook. 

\section{Creating an account on the server}

Please contact Assistant Professor Maurice Cheung for access to the server, users can contact him via email at \textbf{maurice.cheung@yale-nus.edu.sg}. \\

Once a user's access is granted, the user will be given a username and temporary password. To access the server, one can use

\begin{framed}
$\$$ ssh username@172.25.20.52
\end{framed}

Users are encouraged to change their temporary password using the command

\begin{framed}
$\$$ passwd
\end{framed}

\section{Navigating from the command line}
In order to control the server, you will need to know how to navigate from the command line interface. The server is a Linux Redhat system; as such, its commands are consistant with Some basic commands for navigation follow:
\begin{itemize}
\item pwd
\item cd
\item ls
\item mkdir
\item rm
\end{itemize}

\section{Transferring Files to and from Remote Machine}

The \textbf{scp} command can be used to copy files to and from the remote server. To copy files from the local machine into the remote machine, use 

\begin{framed}
$\$$ scp local/path username@172.25.20.52:remote/path
\end{framed}

Similary, to copy files from remote to local, use 

\begin{framed}
$\$$ scp username@172.25.20.52:remote/path local/path 
\end{framed}

NOTE: to copy directories, replace the \textbf{scp} with \textbf{scp -r}.\\

To transfer files on a Windows Machine, one can also install the WinSCP program from winscp.com and use the GUI provided.

\section{Group Account}
Currently, there is also a group account that can be used to share code for others to look at and use. This account has the following details:
\begin{mdframed}
Username: group\\
Password: [on request from Maurice]
\end{mdframed}

\chapter{The Server: Advanced}

\section{Remote Access}

In order to access the server while abroad, it is necessary to use a Virtual Private Network. To access the correct VPN, download and install the forticlient program. The remote gateway SoC VPN can be found at: \textit{webvpn.comp.nus.edu.sg}.\\

\section{Running local files remotely}

You may desire to run local scripts remotely. Unfortunately, local scripts will not be interactive when they are run remotely. For this reason, we recommend copying the files to the remote machine before running. However, it is possible to run local files on a remote machine. The simplest method is to pipe any script on the local machine to the remote machine (replace python with the desired program, if you don't want to execute the script using python):

\begin{mdframed}
$>>>$ \quad cat ~/path/to/file | ssh \textit{username}@172.25.20.52 python -
\end{mdframed}

\section{Continuous Access Using Screen}
In the event that it is necessary to run a extended calculation, you will make use of a powerful program called screen. To start screen, simply type:

\begin{mdframed}
\textgreater\textgreater\textgreater \quad screen
\end{mdframed}

Screen is used to run multiple sessions over the same ssh connection. More importantly, screen allows the user to run commands while disconnected from the server. Screen has an extensive navigational syntax; to learn more:

\begin{mdframed}
press ctrl + a, then ?
\end{mdframed}

Run processes as normal. When you wish to detach from screen:

\begin{mdframed}
press ctrl + a, then d
\end{mdframed}

To reattach:

\begin{mdframed}
\textgreater\textgreater\textgreater \quad screen -r
\end{mdframed}

To lock your session, use:

\begin{mdframed}
press ctrl + a, then x
\end{mdframed}

To terminate your screen session:
\begin{mdframed}
press ctrl + a, then k
\end{mdframed}

To terminate a detached screen session:
\begin{mdframed}
screen -X -S [session \# you want to kill] kill
\end{mdframed}

To list session numbers of active sessions:
\begin{mdframed}
screen -ls
\end{mdframed}

\section{Graphic User Interface}

Download the GUI from https://www.realvnc.com/download/viewer/.\\

\section{Setting up Jupyter notebook on Server}
User can also use Jupyter notebook in the server using the following steps.\\

First, once logged into the server, activate Jupyter Notebook in the server using

\begin{framed}
$\$$ jupyter notebook --no-browser --port=8889
\end{framed}

This opens a Jupyter notebook session in the \textbf{remote} machine at port 8889. Since the remote machine has no display, we need to redirect the display to the local machine. We do that by running the following command \textbf{from the local machine}

\begin{framed}
$\$$ ssh -N -f -L localhost:8888:localhost:8889 username@172.25.20.52
\end{framed}

This command redirects whatever is happening in localhost:8889 of the remote machine to the localhost:8888 of the local machine. \\

Hence, to open the Jupyter notebook session in the server, simply open any browser and access

\begin{framed}
$\$$ localhost:8888
\end{framed}

If this is your first time connecting to the server, your browser will prompt you to provide it with a token. You can find it in the terminal connected to the remote computer.

NOTE: When many users are using the server at the same time port 8889 in the remote machine may not be available, in this case users may try displaying Jupyter notebook from other ports, e.g. you can try ports 8880-8889.


\section{Git and GitHub}
Git is a remote version control system used to backup projects and prevent undue risk when making individual changes to group projects. The Computational Biology git is currently being hosted on https://github.com/mauriceccy/scobra. In order to begin uploading files to the remote repository, you need to initiate a local instance of the repository, sync this repository with the remote repository and gain permission to upload to the master branch.

\subsection{Setting up a local branch and syncing with the remote repository}

If you did not set up your python directory using command line, it is necessary to initiate a repository on your local machine. If you installed scobra using the command line, skip step two and simply sync with the remote repository.

\begin{mdframed}
$>>>$ \quad cd path/to/scobra/directory\\
$>>>$ \quad git init\\
$>>>$ \quad git add .\\
$>>>$ \quad git remote add origin https://github.com/mauriceccy/scobra\\
$>>>$ \quad git remote pull origin master
\end{mdframed}

This will prompt a merge between your local repository and the remote repository, keeping you up to date with the master branch online.

\subsection{Syncing with git}

In rare cases, it may be necessary to add files created on your local machine to the remote repository. In this case, there are two options: you may either commit directly to the master branch, updating the whole git project, or you may commit to a remote branch, which can be merged with the master branch at a later time.\\

To create a new branch on your local machine, use
\begin{mdframed}
$>>>$ \quad git checkout -b localbranchname\\
$>>>$ \quad git remote add remotebranchname
\end{mdframed}

Now you can push to git. To push to your own branch, replace master with your local branch name, and origin with your remote branch name.

\begin{mdframed}
$>>>$ \quad git add .\\
$>>>$ \quad git commit -m "Your commit message"\\
$>>>$ \quad git push origin master
\end{mdframed}







