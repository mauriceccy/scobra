\chapter{Flux Analysis}

\section{Flux Balance Analysis (FBA)}

Given the constraints and objectives, scobra models can be used to simulate metabolism using flux balance analysis. For the sake of simplicity, all simulations here use the simple toy model shown in this figure:

\begin{figure}[h]
\centering
\includegraphics[height=4.5cm, width=5.5cm]{./img/Flux_analysis/scr_res.pdf}
\caption{Simple model}
\end{figure}

%----------------------------------------------------------%

\subsection{Solving the Linear Programming (LP) problem}
The constraints based LP problem can be solved using \textbf{Solve()}. For this example, the method returns an optimal growth rate, i.e., a feasible solution. For problems which are not possible to solve, \textbf{Solve()} will give return "infeasible":

\begin{framed}
$>>>$ m.Solve()\\
optimal
\end{framed}

%----------------------------------------------------------%

\subsection{Solutions}
To get the flux solution as a dictionary use \textbf{GetSol}.

\begin{framed}
$>>>$ m.GetSol()\\
\{'R4': 10.0, 'R5': 3.0, 'R7': 10.0, 'R1': 10.0, 'R2': 13.0, $\cdots$\}
\end{framed}

%----------------------------------------------------------%

\subsection{Printing the solution}
\label{sec:print}

Flux distributions can also be obtained using \textbf{PrintSol}. For our example, the maximum growth rate (maximum possible flux under the input flux assumptions) for R7 and R8 are 10.0 and 3.0, respectively.

\begin{framed}
$>>>$ m.PrintSol()\\
R3: 19.0\\
R2: 13.0\\
R7: 10.0\\
R4: 10.0\\
R1: 10.0\\
R8: 3.0\\
R5: 3.0
\end{framed}

%----------------------------------------------------------%

\subsection{Fixing the flux}

Flux through a reaction can be fixed to a specific value using \textbf{SetFixedFlux}.

\begin{framed}
$>>>$ m.SetFixedFlux(\{'R1':0\})
\end{framed}

%----------------------------------------------------------%

\subsection{Get sum of fluxes}

Get the sum of fluxes for a metabolite using \textbf{FluxSum(}\textit{Metabolite}\textbf{)}.

\begin{framed}
$>>>$ m.FluxSum('E')\\
3.0\\

$>>>$ m.FluxSum('B')\\
13.0\\

$>>>$ m.FluxSum('C')\\ %Note the Flux obtained for `C'.
19.0
\end{framed}

%----------------------------------------------------------%

\subsection{Get constraints}

In order to manipulate the behaviour of the plants, certain constraints are applied which limit the direction and extent of a specific reaction. Use \textbf{GetConstraint(s)} to view these constraints.

\begin{framed}
$>>>$ m.GetConstraint('R1')\\
(0, 10)\\

$>>>$ m.GetConstraints()\\
\{'R4': (0.0, 1000.0), 'R5': (-1000.0, 1000.0), 'R6': (0.0, 1000.0), $\cdots$\}
\end{framed}

%----------------------------------------------------------%

\subsection{Summary information}
The basic information of a solution is obtained using the \textbf{GetState} method:

\begin{framed}
$>>>$ m.GetState()\\
\{'objective\_direction': 'maximize',\\
'solver': None,\\
'bounds': 1000.0,\\
'solution': $<$Solution 13.00 at 0xafdc7cc$>$,\\ 
'objective': \{'R4': 0.0, 'R5': 0.0, 'R6': 0.0, 'R7': 1, 'R1': 0.0, 'R2': 0.0, 'R3': 0.0, 'R8': 1\},\\
'quaduatic\_component': None,\\
'constraints': \{'R4': (0.0, 1000.0), 'R5': (-1000.0, 1000.0), $\cdots$\}\}
\end{framed}

%----------------------------------------------------------%

\subsection{Get flux range}

\begin{framed}
$>>>$ m.AllFluxRange()\\
\{'R4': (0.0, 10.0), 'R5': (0.0, 10.0), 'R6': (0.0, 0.0), 'R7': (0.0, 10.0), 'R1': (0.0, 10.0), 'R2': (0.0, 13.0), 'R3': (0.0, 30.0), 'R8': (0.0, 10.0)\}\\

$>>>$ m=scobra.Model('/home/user/toy.xml')\\
$>>>$ m.AllFluxRange()\\
\{'R4': (0.0, 1000.0), 'R5': (0.0, 333.33), 'R6': (0.0, 0.0), $\cdots$\}
\end{framed}

Note that the flux range is not equivalent to the constraints:

\begin{framed}
$>>>$ m.GetConstraints()\\
\{'R4': (0.0, 1000.0), 'R5': (-1000.0, 1000.0), 'R6': (0.0, 1000.0), $\cdots$\}
\end{framed}

%----------------------------------------------------------%

\section{Further Analysis}

\subsection{Flux Variability Analysis (FVA)}

When scobra returns a flux solution for a given model and set of parameters, this solution is rarely unique. Rather, each reaction can sustain a variety of fluxes as the other fluxes vary. The variation of possible fluxes through a reaction can be important information regarding the importance of different reactions within a particular solution: smaller variation suggests greater importance.\\

In order to determine these different fluxes, we use the method \textbf{FVA}, with arguments for the reactions for which you wish to see the flux variation (as a list) and the number of processes you would like to run.\\

The example below returns flux variability for the first 3 reactions in the model \textit model. It uses four processors to run the computation:

\begin{framed}
$>>>$ model.FVA(model.Reactions()[0:3], processes=4)
\end{framed}

The second example returns flux variability for the reactions in the list, using two processors:

\begin{framed}
$>>>$ model.FVA(reaclist=['R4', 'R5'], processes=2)
\end{framed}

%----------------------------------------------------------%





