\chapter{Getting Started}

\section{Installing Python and PIP using Conda}

Conda is an open source, cross-platform package manager and environment management system that helps with software installation and updating dependencies. \\

What makes conda so powerful is that it works to install packages using dependencies already on your computer. Further, conda keeps track of your packages for you in a special folder, so it is less likely that the \$PATH variable fails. Finally, conda makes uninstalling very easy in cases something goes wrong. \\

To install conda, navigate to "https://docs.conda.io/projects/conda/en/latest/user-guide/install/" and follow the instructions.\\

To install python and pip, use \\

\begin{mdframed}
\textgreater \textgreater\textgreater \quad conda install python
\end{mdframed}

on the terminal in OSX and Linux Machines. On Windows computers, the same command can be run on Anaconda Prompt/Anaconda Powershell Prompt, which can be found from the windows search box.

%-----------------------------------------------------%

%\subsection{Auto-installation}

%As of September 18th, 2016, code is available on the Computational and Systems Biology group for the auto-installation of the scobra package. This code has been tested for Windows 10 systems.

%-----------------------------------------------------%

\subsection{Manual installation}

Scobra can also be installed manually:
\begin{enumerate}
\item Acquire the scobra directory from \url{https://github.com/mauriceccy/scobra} by direct download or through \textbf{git clone \\https://github.com/mauriceccy/scobra}.
\item Place the directory in the location you intend to keep it in. If you place the package in your home directory, you can skip the next step.
\item Update the path variable in python using the\\
\textbf{sys.path.append(}\textit{path/to/scobrapy}) command.
\end{enumerate}

For example, here is a possible scobra installation (note: bash prompts use \$, while python prompts are denoted $>>>$).
\begin{framed}
$\$$ cd Home/Computational\_biology\\
$\$$ git clone https://github.com/mauriceccy/scobra\\
$\$$ python\\
$>>>$ import sys\\
$>>>$ sys.path.append('Home/Computational\_biology')
\end{framed}

Now, when you wish to use scobra in your code, import it as you would any other package:

\begin{framed}
$>>>$ import scobra\\
$>>>$ from scobra.io import * [import all modules from subpackage io]
\end{framed}

%-----------------------------------------------------%

\section{Installing Python Packages using PIP}

Now that python is set up and pip is installed, you can begin to install the various packages you will be using for modelling. \\

\subsection{Setting up a Virtual Environment}
We recommend installing these packages in a virtual environment, to prevent problems that may arise with projects needing different version of the same packages/modules. \\

To set up a conda virtual environment, run 
\begin{framed}
$\$$ conda create --name ENVIRONMENT\_NAME
\end{framed}

To activate the virtual environment, run
\begin{framed}
$\$$ conda activate ENVIRONMENT\_NAME
\end{framed}

You can check that you are in the correct environment by running: 
\begin{framed}
$\$$ conda info --envs 
\end{framed}

The environment with an asterisk next to it is the environment you are in.\\

\subsection{Installing requirements with pip}

Once in the virtual environment, install the requirements needed by scobra using \\

\begin{framed}
$\$$ pip install -r scobra/requirements.txt 
\end{framed}

This effectively clones the environment we use for software development on the users' virtual machine. 

%-----------------------------------------------------%

\section{Updates}

Scobra packages are updated on occasion with bug fixes or additional packages. In order to maintain your local copy of scobra, it is recommended that you regularly check for updates.\\

If you downloaded the scobra package using the automatic installation, or if you downloaded the file manually, you will need to download the updated scobra and overwrite your local version.\\

If you downloaded scobra using git, you can take advantage of git branch management to update your local copy of scobra in place. To do so:

\begin{enumerate}
\item Navigate to the scobra directory using your command line interface.
\item Use \textbf{git status} to check for updates.
\item Run \textbf{git pull} to update with the master branch.
\end{enumerate}

Note: If you have made changes to the scobra directory locally, \textbf{git pull} may return a merge conflict. In this case, consult with the git documentation (\url{https://git-scm.com/documentation}) to resolve the merge conflict, then run the code above as normal.

%-----------------------------------------------------%

\section{Using Jupyter}

Once you have installed IPython/Jupyter, it is useful to learn some of its basics, as it can be very powerful when used well. In particular, Jupyter has its own set of shortcuts:

\begin{enumerate}
\item \textbf{shift + tab} will show the docstring of an object just typed
\item \textbf{ctrl + shift + -} will split the current cell at the cursor
\item \textbf{esc + 0} toggles output
\item \textbf{\%cd new/working/directory} changes the working directory
\item \textbf{!command} executes shell commands
\item \textbf{\%\%kernel} runs the code in that cell using a different kernel (e.g. \textbf{\%\%ruby} runs that cell in ruby)
\end{enumerate}

It also has a command mode, which allows the speedy modification of code:

\begin{enumerate}
\item The \textbf{esc} key will bring you to editor mode. 
\item Directional keys navigate
\item \textbf{b} will add a new cell below
\item \textbf{a} will add a new cell above
\item \textbf{dd} will delete the current cell
\item \textbf{m} will change the current cell to markup mode
\item \textbf{y} will change the current cell to code
\item \textbf{enter} will return you to edit mode
\end{enumerate}

%-----------------------------------------------------%

\section{Using the server}

While it is convenient to have a version locally installed on every user's machine, we sometimes encounter difficulties with installing scobra on some machines. This is one of the reasons we provide users access to a remote server. \\

Apart from that, users who want to run heavier or parallel computations can also utilise this resource. \\ 

The only drawback to working on the server is that users need to be connected to the NUS network, which they need to establish a VPN into the NUS network when working off-campus.\\ 

For instructions on how to access the server and to run Jupyter on server, please refer to Section 4. 

